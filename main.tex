\documentclass[a4paper, titlepage]{article}

\usepackage[utf8]{inputenc}
\usepackage[ngerman]{babel}

\title{Strategiepapier Alte Moschti}
\author{Sascha Huber, Aaron Stampa, Damien Flury}
\date{07. September 2018}

\begin{document}
\maketitle

\section{Anspruchsgruppen}
\subsection{Kundinnen und Kunden}
\begin{itemize}
  \item Günstige Eintrittspreise, akzeptable Preise.
  \item Lokale Musiker
\end{itemize}
\subsection{Eigenkaptialgeber/Eigentümer}
Mit dem Gewinn möchten wir die Dienstleistungen verbessern, d.h. weitere Musiker engangieren,
Werbekampagnen starten, etc. Als Genossenschaft möchten wir unseren privaten Reichtum von der
alten Moschti trennen und somit Sicherheit gewähren-
\subsection{Fremdkapitalgeber}
Gut platzierte, sichtbare Werbung für Sponsoren, möglichst regelmässige Zeiten, stetige Kundschaft.
\subsection{Mitarbeiternde}
Fairer, nicht hoher Lohn. Gute Arbeitsatmosphäre.
\subsection{Lieferanten}
Langfristige Beziehungen, pünktliche Bezahlung, Werbung für Lieferanten.
\subsection{Institutionen}
Umweltgerechtes Verhalten, Müll trennen, gute Arbeitsbedingungen,Sponsoring
\subsection{Staat}
Steuereinnahmen, Einhalten von Gesetzen, langfristige Arbeitsstellen
\subsection{Konkurrenz}
Fairer Wettkampf, gegenseitige Unterstützung

\section{Umweltsphären}
\subsection{Ökonomische Umweltsphäre}
Ökonomische Risiken sind plötzliche Wirtschaftskrien, Inflation, sinkende Arbeitslöhne.
\subsection{Technologische Umweltsphäre}
Neue Techno-Partys in der Umgebung.
\subsection{Ökologische Umweltsphäre}
Erdrutsche verhinderin Zufahrt zur Moschti.
\subsection{Soziale Umweltsphäre}

\end{document}