\documentclass[a4paper, titlepage]{article}

\usepackage[utf8]{inputenc}
\usepackage[ngerman]{babel}

\title{Strategiepapier Alte Moschti}
\author{Sascha Huber, Aaron Stampa, Damien Flury}
\date{07. September 2018}

\begin{document}
\maketitle

\tableofcontents
\newpage

\section{Anspruchsgruppen}
\subsection{Kundinnen und Kunden}
\begin{itemize}
  \item Günstige Eintrittspreise
  \item Gute Preise für Verpflegung
  \item Gute Musiker
\end{itemize}
\subsection{Eigenkaptialgeber/Eigentümer}
Mit dem Gewinn möchten wir die Dienstleistungen verbessern, d.h. weitere Musiker engangieren,
Werbekampagnen starten, etc. Als Genossenschaft möchten wir unseren privaten Reichtum von der
alten Moschti trennen und somit Sicherheit gewähren-
\subsection{Fremdkapitalgeber}
Gut platzierte, sichtbare Werbung für Sponsoren, möglichst regelmässige Zeiten, stetige Kundschaft.
\subsection{Mitarbeiternde}
Fairer, nicht hoher Lohn. Gute Arbeitsatmosphäre.
\subsection{Lieferanten}
Langfristige Beziehungen, pünktliche Bezahlung, Werbung für Lieferanten.
\subsection{Institutionen}
Umweltgerechtes Verhalten, Müll trennen, gute Arbeitsbedingungen,Sponsoring
\subsection{Staat}
Steuereinnahmen, Einhalten von Gesetzen, langfristige Arbeitsstellen
\subsection{Konkurrenz}
Fairer Wettkampf, gegenseitige Unterstützung

\section{Umweltsphären}
\subsection{Ökonomische Umweltsphäre}
Ökonomische Risiken sind plötzliche Wirtschaftskrien, Inflation, sinkende Arbeitslöhne.
\subsection{Technologische Umweltsphäre}
Neue Techno-Partys in der Umgebung.
\subsection{Ökologische Umweltsphäre}
Verhinderung der Zufahrt zur Moschti aufgrund Strassensperren, etc.
\subsection{Soziale Umweltsphäre}
Rauchverbot in der Moschti, erlaubt ausserhalb. 
\section{Zentrale Werte}
\begin{itemize}
  \item Unterstützung unbekannter Musiker
  \item Für die Gesellschaft
  \item Musik für Jung und Alt
  \item Schutz der Umwelt
\end{itemize}
\section{Vision}
Möglichst viele Menschen mit der Moschti erreichen, unterhalten und die Gewinne maximieren. 
\section{Öffentliches Leitbild}
Wir sind daran interessier unsere Leitsungen stets zu verbessern und somit allen Kunden gerecht zu werden.
\section{SWOT-Analyse}
\subsection{Stärken (Strenghts)}
\begin{itemize}
  \item Treue Kunden
  \item Langfristige Sponsoren
  \item Qualitativ gute Musiker
  \item Gutes Preisleitungsverhältnis
  \item Genügend Parkplätze
\end{itemize}
\subsection{Schwächen (Weaknesses)}
\begin{itemize}
  \item Abgelegener Ort
  \item Schlechte ÖV-Verbindung
  \item Unglückliche Nachbaren
\end{itemize}
\subsection{Möglichkeiten (Opportunities)}
\end{document}